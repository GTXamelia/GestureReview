\chapter{Conclusions}
We are moving further and further away from traditional methods of interacting with computers. We've gone from having to use dials and wires, to using a keyboard and mouse. Between these two stages there was a slow progression to make interacting with a computers easier and require less and less knowledge of how a computer works. Gesture-based technology is the next step in human-machine interaction, the questions isn't if it'll happen, but when. In its current state gesture based technology has faced issues with developer adoption, which ultimately leads to consumer dissatisfaction with the product. 

Gestures as a communication tool is one of the oldest methods of communication and will remain as long as two people can communicate with one another. The limitations in gesture technology in the intent and context behind the gesture. Human language requires context and is full of innuendos, which carries over to gesture-based communication. These small innuendos are hard to teach machines, so in the current generation workaround need to be used in order to fix this small problem.

Ultimately gesture controls will replace interaction with a computer for most people. The challenge comes with adoption rate with physical gesture as voice gestures are being implemented into device users are already familiar with and have such as smartphones.